\section{Licencias}
\frame
{
	\frametitle{Licencias}
	\begin{center}
	Licencias
	\end{center}
}

\subsection{Que son las licencias?}
\frame
{
	\frametitle{Que son los derechos de autor?}
	\begin{itemize}
	\item bla bla bla
	\item bla bla bla
	\end{itemize}
}

\subsection{tipos licencias}
\frame
{
	\frametitle{Las licencias bla}
	\begin{itemize}
	\item bla bla bla
	\item  bla bla bla
	\end{itemize}
}

\subsection{"Pirateria" de software}
\frame
{
	\frametitle{Porque "pirata"?}
	\begin{itemize}
	\item bla bla bla
	\item bla bla bla
	\end{itemize}
}

\subsection{Clasificacion}
\frame
{
	\frametitle{Clasificacion}
	\begin{itemize}
	\item Comercial
	\item shareware
	\item Freeware (libre?)
	\item Dominio publico
	\end{itemize}
	
	\frametitle{Alcance}
	\begin{itemize}
	\item De maquina
	\item De uso personal 
	\item De uso concurrente
	\item De ubicacion o corporativa
	\end{itemize}
}


\subsection{Dos ejemplos concretos}
\frame
{
	\frametitle{MS-EULA (End User Licence Agreement}
	\begin{itemize}
	\item Se prohibe la copia
	\item El software solo puede ser empleado en un �nico ordenador con un m�ximo de 2 procesadores
	\item No puede ser empleado como webserver o fileserver.
	\item Puede dejar de funcionar si se efect�an cambios en el hardware.
	\item Las actualizaciones del sistema pueden modificar la EULA.
	\item Solo puede ser transferida una vez a otro usuario.
	\item Limita la ingenier�a inversa.
	\item S� podr� en cualquier momento recoger informaci�n del sistema y su uso, la cual podr� suministrar dicha informaci�n a terceros.
	\item La garant�a tan solo cubre los primeros 90 d�as.
	\item Actualizaciones y parches sin garant�a.
	\end{itemize}
	
	\frametitle{Alcance}
	\begin{itemize}
	\item Permite la copia, modificaci�n y redistribuci�n del software.
	\item Garant�a de los derechos del ciudadano a la copia, modificaci�n y redistribuci�n del software.
	\item No ofrece garant�as sobre el producto.
	\item Puede ser vendido y se puede cobrar por los servicios sobre el software.
	\item Cualquier patente sobre el mismo debe ser licenciada para el beneficio de todos.
	\item Debe incluir el c�digo fuente.
	\end{itemize}
}
