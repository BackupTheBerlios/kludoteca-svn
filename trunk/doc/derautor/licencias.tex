\section{Licencias}
\frame
{
	\frametitle{Licencias}
	\begin{center}
	Licencias
	\end{center}
}

\subsection{Que son las licencias?}
\frame
{
% \documentclass[10pt]{beamer}
% 
% \usepackage[latin1]{inputenc}
% \usepackage{beamerthemebars}
% 
% \usepackage[ngerman]{babel}
% \usepackage[T1]{fontenc}
% \usepackage[dvips]{graphicx}
% 
% \date{21/11/05}
	\frametitle{Licencias}
	\begin{itemize}
	\item Una \textbf{Licencia de software} (en ingles software license) es la autorizacion o permiso concedida por el titular del derecho de autor, en cualquier forma contractual, al usuario de un programa informatico, para utilizar este en una forma determinada y de conformidad con unas condiciones convenidas.

La licencia, que puede ser gratuita u onerosa, precisa los derechos (de uso, modificaci�n y/o redistribucion) concedidos a la persona autorizada y sus limites. Ademas, puede se�alar el plazo de duraci�n, el territorio de aplicacion y todas las demas cl�usulas que el titular del derecho de autor establezca.
	\end{itemize}

}

\subsection{Tipos Licencias}
\frame
{
	\frametitle{Tipos}
	\begin{itemize}
	\item \textbf{Derechos que cada autor}
		\begin{itemize}
		\item Licencia de software libre sin protecci�n heredada
		
		\item Se puede crear una obra derivada sin que esta tenga obligaci�n de protecci�n alguna. Muchas licencias pertenecen a esta clase, entre otras:
		\item BSD License.
		\item MIT License.
		
		\end{itemize}

		
	\item \textbf{Segun su destinatario}
		\begin{itemize}
		\item Licencia de Usuario FinalEn ingl�s EULA o end user license agreement, es una licencia en que se permite s�lo el uso del mismo.
		\end{itemize}
	\end{itemize}
}

\subsection{"Pirateria" de software}
\frame
{
	\frametitle{Porque "pirata"?}
	\begin{itemize}
	\item bla bla bla
	\item bla bla bla
	\end{itemize}
}

\subsection{Clasificacion}
\frame
{
	\frametitle{Clasificacion}
	\begin{itemize}
	\item Comercial
	\item shareware
	\item Freeware (libre?)
	\item Dominio publico
	\end{itemize}
	
	\frametitle{Alcance}
	\begin{itemize}
	\item De maquina
	\item De uso personal 
	\item De uso concurrente
	\item De ubicacion o corporativa
	\end{itemize}
}


\subsection{Dos ejemplos concretos}
\frame
{
	\frametitle{MS-EULA (End User Licence Agreement}
	\begin{itemize}
	\item Se prohibe la copia
	\item El software solo puede ser empleado en un �nico ordenador con un m�ximo de 2 procesadores
	\item No puede ser empleado como webserver o fileserver.
	\item Puede dejar de funcionar si se efect�an cambios en el hardware.
	\item Las actualizaciones del sistema pueden modificar la EULA.
	\item Solo puede ser transferida una vez a otro usuario.
	\item Limita la ingenier�a inversa.
	\item S� podr� en cualquier momento recoger informaci�n del sistema y su uso, la cual podr� suministrar dicha informaci�n a terceros.
	\item La garant�a tan solo cubre los primeros 90 d�as.
	\item Actualizaciones y parches sin garant�a.
	\end{itemize}
	
	\frametitle{Alcance}
	\begin{itemize}
	\item Permite la copia, modificaci�n y redistribuci�n del software.
	\item Garant�a de los derechos del ciudadano a la copia, modificaci�n y redistribuci�n del software.
	\item No ofrece garant�as sobre el producto.
	\item Puede ser vendido y se puede cobrar por los servicios sobre el software.
	\item Cualquier patente sobre el mismo debe ser licenciada para el beneficio de todos.
	\item Debe incluir el c�digo fuente.
	\end{itemize}
}
